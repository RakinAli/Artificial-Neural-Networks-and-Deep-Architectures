\documentclass[a4paper]{article}

\usepackage[swedish]{babel}
\usepackage[latin1]{inputenc}
\usepackage{amssymb}
\usepackage{framed}

\setlength{\parindent}{0pt}
\setlength{\parskip}{3ex}

\begin{document}

\begin{center}
  {\large Artificial Neural Networks and Deep Architectures, DD2437}\\
  \vspace{7mm}
  {\huge Short report on lab assignment 3\\[1ex]}
  {\Large Hopfield Networks}\\
  \vspace{8mm}  
  {\Large Rakin Ali, Steinar Logi and Hassan Alzubeidi\\}
  \vspace{4mm}
  {\large February 14, 2024\\}
\end{center}

\section{Main objectives and scope of the assignment}
Our major goals in the assignment were  
\begin{itemize}
\item to build a Hopfield network from scratch only using Python and theory from the lab instructions.
\item to explore associative memory capabilities of a Hopfield network.
\end{itemize}

The weights of the Hopfield network were adjusted to the data points, they were not manually created.

\section{Methods}
All code was created with Python v 3.9 and Jupyter notebook. This assignment was difficult to break into different parts so all instructions were completed three times thereafter we compared our results in order to check if our networks were behaving the way they were suppose to. 

\section{Results and discussion}
This section presents the results provides a small discussion which analyzes the results.

\subsection{Convergence and attractors}
Given x1d, x2d and x3d where x1d has one bit error while x2d and x3d has two bits errors, we discovered that the network converged towards to the stored patterns. When we made the starting pattern even more dissimilar to the stored one where more than half the points were wrong, it did not converge correctly. 

\subsection{Sequential update}
All three patterns were stable and tested with the network. The network was able to memorise them. \\
\\Later when the network tried patterns p10 and p11, interesting results were noted. P10 was handled well while P11 did not. P11 seems to be spurious. \\

\subsection{Energy}


\subsection{Distortion Resistance}

\subsection{Capacity}

\subsection{Sparse Pattern}

\section{Final remarks \normalsize{\textit{(max 0.5 page)}}}
\textit{Please share your final reflections on the lab, its content and your own learning. Which parts of the lab assignment did you find confusing or not necessarily helping in understanding important concepts and which parts you have found interesting and relevant to your learning experience? \\
Here you can also formulate your opinion, interpretation or speculation about some of the simulation outcomes. Please add any follow-up questions that you might have regarding the lab tasks and the results you have produced.}

\end{document}
